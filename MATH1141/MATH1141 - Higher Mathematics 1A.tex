\documentclass{article}

% Packages
\usepackage{import}
\usepackage{xcolor}
\usepackage{cancel}
\usepackage{mathtools}
\usepackage{hyperref}
\usepackage{setspace}
\usepackage{float}

% Custom perms and combs commands
\newcommand\Myperm[2][^n]{\prescript{#1\mkern-2.5mu}{}P_{#2}}
\newcommand\Mycomb[2]{\prescript{#1\mkern-0.5mu}{}C_{#2}}

% Colors
\definecolor{yorhabg}{HTML}{131314}
\definecolor{yorhafg}{HTML}{C9C7CD}
\definecolor{yorhagrid}{HTML}{B5AF9C}
\definecolor{mred}{HTML}{D67069}
\definecolor{mblue}{HTML}{6887A1}

\usepackage[dvipsnames]{xcolor}
\definecolor{pastel_red}{RGB}{255, 173, 173}
\definecolor{pastel_orange}{RGB}{255, 214, 165}
\definecolor{pastel_yellow}{RGB}{253, 255, 182}
\definecolor{pastel_green}{RGB}{176, 217, 176}
\definecolor{pastel_blue}{RGB}{167, 199, 231}
\definecolor{pastel_purple}{RGB}{222, 218, 244}

\pagecolor{yorhabg}		% Set background color
\color{yorhafg}

\import{preamble.sty}

\usepackage[T1]{fontenc}			% Font package

\usepackage{fouriernc}

\usepackage{sectsty}
\usepackage{graphicx}
\usepackage{amsmath}
\usepackage{amsfonts}
\usepackage{amssymb}
\usepackage[skins, most]{tcolorbox}

\DeclareMathOperator{\sgn}{sgn}

\usepackage{tikz}
\usepackage{eso-pic}
\usetikzlibrary{calc, shadows.blur}
\usetikzlibrary{angles, quotes}
\usetikzlibrary{3d}

% Margins
\topmargin=0in
\evensidemargin=0in
\oddsidemargin=0in
\textwidth=6.5in
\textheight=9.0in
\headsep=0.25in

\AtBeginEnvironment{tcolorbox}{\small}

\newtcolorbox{imp}{enhanced,arc=0mm,colback=yorhabg,colframe=mred,leftrule=10mm,coltext=yorhafg,%
overlay={\node[anchor=west,outer sep=2pt] at (frame.west) {\includegraphics[width=6mm]{images/imageb.png}}; }}

\newtcolorbox{shortcut}{enhanced,arc=0mm,colback=yorhabg,colframe=mred,leftrule=10mm,coltext=yorhafg, coltitle=yorhabg, title=\texttt{Shortcut.}, 
overlay={\node[anchor=west,outer sep=2pt] at (frame.west) {\includegraphics[width=6mm]{images/imageb.png}}; }}

\newtcolorbox{question}[1]{
    enhanced, 
    colback=yorhabg,
    colframe=mblue,
    coltext=yorhafg,
    coltitle=yorhabg,
    attach boxed title to top left={yshift*=-\tcboxedtitleheight}, 
    title=\texttt{#1},
    boxed title size=title,
    boxed title style={%
        rounded corners=northeast, 
        rounded corners=northwest, 
        colback=tcbcolframe, 
        boxrule=0pt,
    },
    underlay boxed title={%
        \path[fill=tcbcolframe] (title.south west)--(title.south east) 
            to[out=0, in=180] ([xshift=5mm]title.east)--
            (title.center-|frame.east)
            [rounded corners=5pt] |- 
            (frame.north) -| cycle; 
    },
}

\newcommand\bb[1]{\textcolor{yorhafg}{\textbf{#1}}}

\title{\textbf{MATH1141: Higher Mathematics 1A}}
\author{L. Cheung}
\date{\today}

\begin{document}

\maketitle
\tableofcontents
\newpage

\chapter{Calculus}
\section{Content Overview}
	\begin{itemize}
		\item Functions: continuity, differentiability, invertibility (inverse functions)
		\item Curve sketching
		\item Integration: area, Riemann sum (approximation), Fundamental theorems of calculus
		\item Logarithm: exponential, hyperbolic
	\end{itemize}

\section{Sets, inequalities and functions}
	\subsection{Sets of numbers}
		A \textcolor{pastel_blue}{set} is a collection of distinct objects. The objects in a set are elements or members of the set. Commonly used sets:
			\begin{itemize}
				\item The set of natural numbers: $\mathbb{N} = \{0, 1, 2, 3, 4,...\}$
				\item The set of integers: $\mathbb{Z} = \bigl\{...,-3,-2,-1,0,1,2,3,...\}$
				\item The set of rational numbers: $\mathbb{Q} = \{\frac{p}{q}: p,q \text{ are integers and } q \neq 0\}$
				\item The set of $\mathbb{R}$ real numbers may be represented as the collection of points lying on the number line
			\end{itemize}
		If $A$ is a set of numbers and the number $x$ is a member of the set $A$, then we write:
			\begin{align*}
				x \in A
			\end{align*}

		If $x$ is not a member of $A$, then we write
			\begin{align*}
				x \notin A
			\end{align*}


	\subsection{Intervals}
		\begin{itemize}
			\item Parenthesis: excludes endpoints
			\item Bracket: includes endpoints
			\item Combination: neither open nor closed interval
		\end{itemize}

	\subsection{Solving Inequalities}
		For $x, y, z \in \mathbb{R}$:
			\begin{itemize}
				\item if $x>y$, then $x+z>y+z$ 
				\item if $x>y$ and $z>0$, then $xz>yz$
				\item if $x>y$ and $z<0$, then $xz<yz$
			\end{itemize}

		\begin{question}{Example}
			Solve the quadratic inequality $x^2 + 4x > 21$
			\begin{align*}
				x^2 + 4x > 21 & \Longleftrightarrow x^2+4x-21>0 \\
							  & \Longleftrightarrow (x+7)(x-3)>0 \\
				\text{Solution set: } & x < -7 \text{ or } x > 3 \\
									  & x \in (-\infty, -7) \cup (3, \infty)			\end{align*}
		\end{question}

		\begin{question}{Example}
			Solve the rational inequality $\frac{1}{x+1}<\frac{1}{(x-2)(x-2)}$
			\begin{align*}
				\text{Multiply both sides by } & (x+1)^2(x-2)^2(x-3)^2 \quad  (>=0), \\
				\frac{1}{x+1}<\frac{1}{(x-2)(x-3)} & \Longleftrightarrow (x+1)(x-2)^2(x-3)^2 < (x+1)^2(x-2)(x-3) \\
												   & \Longleftrightarrow (x+1)(x-2)^2(x-3)^2 - (x+1)^2(x-2)(x-3) <0 \\
												   & \Longleftrightarrow (x+1)(x-2)(x-3)(x-4)(x-5) < 0 \\
							 \text{Solution set: } & (-\infty, -1) \cup (1,2) \cup (3,5)
			\end{align*}
		\end{question}

		\begin{figure}[H]
			\begin{center}
				\includegraphics[width=0.6\textwidth]{images/fig1}
			\end{center}
		\end{figure}

	\subsection{Absolute Values}
		The absolute value (or magnitude) of a real number $x$ is
			\begin{align*}
				|x| = \left\{
					\begin{aligned}
						x & \quad \text{if } x\leq0, \\
						-x& \quad \text{if } x<0
					\end{aligned}
			\end{align*}

		\begin{question}{Example}
			Solve the inequality $|3x+1| \leq 4$
			\begin{align*}
				|3x+1| \leq 4 & \Longleftrightarrow |x+ \frac{1}{3}| \leq \frac{4}{3} \\
							  & \Longleftrightarrow x + \frac{1}{3} \leq \frac{-4}{3} \quad \text{or} \quad x + \frac{1}{3} \leq \frac{4}{3} \\
							  & \Longleftrightarrow x \leq \frac{-5}{3} \quad \text{or} \quad x \leq 1 \\
				\text{Solution set: } (-\infty, \frac{-5}{3}] \cup [1, \infty)
			\end{align*}
		\end{question}

		\begin{question}{Example}
			Solve the inequality $\frac{|x+5|}{|x-11|} < 1$
			\begin{align*}
				\frac{|x+5|}{|x-11|}<1 & \Longleftrightarrow |x+5| < |x-11| \quad\text{(multiply both sides by |x-11|)} \\
									   & \Longleftrightarrow |x+5|^2 < |x-11|^2 \\
									   & \Longleftrightarrow x^2 + 10x + 25 < x^2 -22x + 121 \\
									   & \Longleftrightarrow 32x < 96 \\
									   & \Longleftrightarrow x < 3 \\
				\text{Solution set: } (-\infty, 3) \\
				\text{\textcolor{pastel_red}{Note:} 11 is not in the solution set}
			\end{align*}
		\end{question}

	\subsection{Functions}
		\subsubsection{Domain, codomain and range}
			Let $A$ and $B$ be subsets of $\mathbb{R}$: A function with domain $A$ and codomain $B$ is a rule which assigns to every $x \in A$ exactly one number $f(x) \in B$
			\begin{align*}
				& f: A \rightarrow B \\ 
				& f: A \ni x \mapsto f(x) \in B
			\end{align*}

			\begin{itemize}
				\item $A = \text{Dom}(f)$: set of all inputs
				\item $B = \text{Codom}(f)$: set that contains all the outputs
				\item Range of $f$: set of actual outputs, defined by:
					\begin{align*}
						\text{Range}(f) = \bigl\{f(x) \in B: x \in A \bigr\}
					\end{align*}
				\item Hence, $\text{Range}(f) \subseteq \text{Codom}(f)
				\item The \textcolor{pastel_red}{codomain does not have to be unique}
				\item The codomain is useful when the range is unknown
			\end{itemize}

			\begin{question}{Example}
				Given: $f: [0,\infty) \rightarrow \mathbb{R}, f(x) = 2+\sqrt{x}, \forall x \in [0,\infty)$ \\
				Then, $\text{Dom}(f) = [0,\infty), \text{ Codom}(f) = \mathbb{R}, \text{ and Range}(f) = [2,\infty)$
			\end{question}

			Natural domain (or maximal domain) of $f$ is the largest possible domain for which the rule makes sense (for real numbers)

	\subsubsection{Forming new functions}
		Let $A$, $B$, $C$ and $D$ be subsets of $\mathbb{R}$. Given two functions $f: A \rightarrow B$ and $g: A \rightarrow B$, then the functions
			\begin{align*}
				f+g,\quad f-g, \quad fg,\quad \displaystyle\frac{f}{g}
			\end{align*}

		are defined by the rules
			\begin{align*}
				(f+g)(x) &= f(x)+g(x),\quad \forall x \in A \\
				(f-g)(x) &= f(x)-g(x), \quad \forall x \in A \\
				(fg)(x) &= f(x)g(x), \quad \forall x \in A \\
				\Bigl(\frac{f}{g}\Bigr)(x) &= \frac{f(x)}{g(x)}, \quad \forall x \in \text{A provided that } g(x) ̸= 0.
			\end{align*}

		To form these new functions, the domains of both $f$ and $g$ must be the same

		Suppose that $f:C \rightarrow D$ and $g:A \rightarrow B$ are functions such that
			\begin{align*}
				\text{Range}(g) \subseteq \text{Dom}(f) = C
			\end{align*}

		Then the composition function $(f \circ g)(x) = f(g(x)),\quad \forall x \in A$

	\subsection{The Elementary Functions}
		Elementary functions are those that can be constructed by combining a finite number of polynomials, exponentials, logarithms, roots, trigonometric and inverse trigonometric functions via function composition ($\circ$), $+$, $-$, $\times$, and $\div$.

	\subsection{Implicitly Defined Functions}
		Many curves on the plane can be described by the points $(x,y)$ that satisfy some equation involving x and y. However, not all these curves are necessarily functions. Some cannot be expressed with a single $y$ term ($y = \dots$). In this case, the curve may be decomposed into two or more functions that are \textcolor{pastel_blue}{implicitly} defined by the curve.

	\subsection{Continuous Functions}
		In the below example, the $\tan{x}$ graph breaks at $x = -\frac{\pi}{2}, \frac{\pi}{2}, \dots$.

		\begin{figure}[H]
			\begin{center}
				\includegraphics[width=0.7\textwidth]{images/tangraph.png}
			\end{center}
		\end{figure}

		However, this \textcolor{pastel_red}{break in the domain does not make it a discontinuity in the graph}. Discontinuities can be categorised as following:

		\begin{figure}[H]
			\begin{center}
				\includegraphics[width=0.95\textwidth]{images/discontinuities.png}
			\end{center}
		\end{figure}
		
\chapter{Algebra}


\end{document}
